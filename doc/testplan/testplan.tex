\documentclass{article}
\usepackage{fullpage}
\usepackage{graphicx}
\usepackage{tabularx}
\newenvironment{testcase}
{
  \tabularx{\textwidth}{|p{1.5in}|X|}
  \hline 
  }{
    \hline
    \endtabularx
}

\begin{document}

\title{Acceptance Test Plan for Mashbot} 
\author{George D'Andrea \and Andrew Gall \and Josiah Kiehl \and
  Cody Ray \and Vito Salerno}
\date{\today}
\begin{titlepage}
\maketitle
\end{titlepage}

\section*{Revision History}
\begin{tabular}{|p{2in}|l|l|l|}
  \hline
  \textbf{Name} & \textbf{Date} & \textbf{Reason for Changes} & \textbf{Version} \\
  \hline \hline
  George D'Andrea, Andrew Gall, Josiah Kiehl, Cody Ray, Vito
  Salerno & 1 December 2009 & Initial Version & 1.0 \\
  \hline
\end{tabular}

\clearpage
\tableofcontents
\clearpage

\section{Introduction} % VS

\subsection{Background}

This document describes the battery of tests the Mashbot product will
be required to pass in order to be considered successful. Mashbot is a
tool for managing advertising campaigns and collecting data from
social networks. It is described in detail in the Software
Requirements Specification for Mashbot.

\subsection{Structure of Document}
\begin{itemize}
  \item Section 2 - Describes the overall approach to the Acceptance
    Test Plan
  \item Section 3 - Describes in more detail features covered
    or not covered by the Acceptance Test Plan
  \item Section 4 - Describes the criteria which must be satisfied to
    begin and complete the Acceptance Test Plan
  \item Section 5 - Describes the roles and responsibilities of the
    staff members involved in the Acceptance Test plan and procedures for
    reporting test results and testing problems
  \item Section 6 - Describes the actual test cases in the Acceptance
    Test Plan
\end{itemize}


\subsection{References}
\subsection{Glossary}

\section{Test Approach and Restraints}

\subsection{Introduction}
\subsection{Test Objectives}
\subsection{Test Structure}

\section{Test Assumptions and Exclusions}

\subsection{Introduction}
\subsection{Assumptions}
\subsection{Exclusions}

\section{Entry and Exit Criteria}

\subsection{Introduction}
\subsection{Entry Criteria}
\subsection{Exit Criteria}

\section{Testing Participants}

\subsection{Introduction}
\subsection{Roles and Responsibilities}
\subsection{Training Requirements}
\subsection{Problem Reporting}
\subsection{Progress Reporting}

\section{Testing Project Test Cases}

\subsection{Introduction}
\subsection{Test Cases}
\subsubsection{User Account Creation} % CR form validation 
\subsubsection{User Account Modification} % CR
\subsubsection{Logging In} % CR
\subsubsection{Logging Out} % CR
\subsubsection{Login Timeout} % CR
\subsubsection{Campaign Creation} % JK
\begin{testcase}
  example & example. \\
\end{testcase}
\subsubsection{Campaign Modification} % JK
\subsubsection{Add Campaign Content} % JK
\subsubsection{Modify Campaign Content} % JK
\subsubsection{Content Scheduler} % JK
\subsubsection{Content Publisher} % JK
\subsubsection{Adding External Service Account} % JK

\end{document}


% LocalWords:  Mashbot
