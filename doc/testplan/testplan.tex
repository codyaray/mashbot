\documentclass{article}
\usepackage{fullpage}
\usepackage{graphicx}
\usepackage{tabularx}
\newenvironment{testcase}
{
  \noindent
  \tabularx{\textwidth}{|p{1.5in}|X|}
  \hline 
  }{
    
    \endtabularx
}
\newcommand{\subsubsubsection}[1]{
  \vspace{.5em}
  \noindent
  \textbf{#1} \\*[-.25em]
  \nopagebreak
}
\begin{document}

\pagestyle{plain}
\title{Acceptance Test Plan for Mashbot} 
\author{George D'Andrea \and Andrew Gall \and Josiah Kiehl \and
  Cody Ray \and Vito Salerno}
\date{\today}
\begin{titlepage}
\maketitle
\end{titlepage}

\section*{Revision History}
\begin{tabular}{|p{2in}|l|l|l|}
  \hline
  \textbf{Name} & \textbf{Date} & \textbf{Reason for Changes} & \textbf{Version} \\
  \hline \hline
  George D'Andrea, Andrew Gall, Josiah Kiehl, Cody Ray, Vito
  Salerno & 1 December 2009 & Initial Version & 1.0 \\
  \hline
\end{tabular}

\clearpage
\tableofcontents
\clearpage

\section{Introduction} % VS

\subsection{Background}

This document describes the battery of tests Mashbot will be required
to pass in order to be considered successful. Mashbot is a tool for
managing advertising campaigns and collecting data from social
networks. It is described in detail in the Software Requirements
Specification for Mashbot.

\subsection{Structure of Document}
\begin{itemize}
  \item Section 2 - Describes the overall approach to the Acceptance
    Test Plan
  \item Section 3 - Describes in more detail features covered
    or not covered by the Acceptance Test Plan
  \item Section 4 - Describes the criteria which must be satisfied to
    begin and complete the Acceptance Test Plan
  \item Section 5 - Describes the roles and responsibilities of the
    staff members involved in the Acceptance Test plan and procedures for
    reporting test results and testing problems
  \item Section 6 - Describes the actual test cases in the Acceptance
    Test Plan
\end{itemize}


\subsection{References}
Software Requirements Specification for Mashbot

\subsection{Glossary}

\begin{itemize}
\item Test Team Leader - The person in charge of testing.
\item Project Leader - The person in charge of the project.
\item Software Requirements Specification - A Software Requirements Specification is a document which
  describes the behavior of a system.
\item Functional Requirements - Functional Requirements define the internal workings of the software.
\item Unit Tests - Unit Testing is a testing phase which validates
  that individual modules and other units of source code are working
  properly.
\item Integration Tests - Integration Testing is the phase following
  Unit Testing and tests the connections between units.
\item System Tests - System Tests are conducted after Integration
  Testing to evaluate the system’s compliance with its specified
  requirements.
\end{itemize}

\section{Test Approach and Restraints} % VS

\subsection{Introduction}

This section describes the overall approach, techniques and
testing tools to be used in Acceptance Test Plan for Mashbot and any constraints that may apply.

\subsection{Test Objectives}

The Acceptance Test Plan will examine Mashbot and verify
whether it fulfills the requirements set forth in the Software
Requirements Specification.


\subsection{Test Structure}

The Acceptance Test Plan will consist of a subset of test cases and
methods, previously used in the Unit Tests, Integration Tests and
System Tests for Mashbot. The test cases will be carefully selected to
allow for the verification of the functional requirements of Mashbot
as listed in the Software Requirements Document. It is essential that
all appropriate Unit Tests, Integration Tests and System Tests were
successfully performed prior to the Acceptance Tests.

\section{Test Assumptions and Exclusions} % ND

\subsection{Introduction}

This section describes details about which aspects of Mashbot are
within the scope of this Acceptance Test Plan and which are not.

\subsection{Assumptions}

The Acceptance Test Plan covers:

\begin{itemize}
\item Functional requirements of Mashbot listed in the Software
Requirements Specification
\item Usability of Mashbot
\item Consistency of user documentation for Mashbot
\end{itemize}

\subsection{Exclusions}

The Acceptance Test Plan does not cover:

\begin{itemize}
\item Non-functional requirements of Mashbot besides usability, as listed in
the Software Requirements Specification
\item Quality of code
\end{itemize}

\section{Entry and Exit Criteria} % AG


\subsection{Introduction}
	This section specifies criteria that must be met in order for the Acceptance 
	Test Plan to be executed. Additionally, it specifies the criteria that must 
	be met for success and failure of the Acceptance Test Plan.
\subsection{Entry Criteria}
	The Acceptance Test Plan can be executed when these preconditions are met:
	\begin{itemize}
		\item Mashbot has successfully passed testing including:
			\begin{itemize}
				\item Integration Testing
				\item Systems Testing
				\item Unit Testing
			\end{itemize}
		\item A testing environment has been setup which reflects the system 
		requirements as stated in our Software Requirements Specification.
		\item Copies of the latest versions of both the Software Requirements 
		Specification and other user-related documentation has been received.
		\item The latest released version of Mashbot has been received.
		\item Consent of the following people to begin the Acceptance Test Plan 
		has been given:
			\begin{itemize}
				\item Project Leader
				\item Test Team Leader
			\end{itemize}
	\end{itemize}
\subsection{Exit Criteria}
	A run of the Acceptance Test Plan has several possible outcomes:
	\begin{itemize}
		\item Success: All \textbf{Priority 1} requirements were tested and 
		performed as required.
		\item Failure:
			\begin{itemize}
				\item At least one of our \textbf{Priority 1} requirement 
				deviated from expected behavior.
				\item The Acceptance Test Plan was rescheduled on approval from 
				the Test Team Leader.
			\end{itemize}
		
	\end{itemize}
	

\section{Testing Participants} % ND

\subsection{Introduction}

This section describes the roles and responsibilities of people involved in the
Acceptance Test Plan, as well as how to report results.

\subsection{Roles and Responsibilities}

\begin{description}
\item[Test Team Leader] George D'Andrea
\item[User Representative] A member of the Drexel College of Engineering
faculty or the advisor to this project who will overview the testing process
\item[Tester] A person who will execute the tests
\end{description}

\subsection{Training Requirements}

Everyone involved with the test process should be familiar with Mashbot, its
user interace, the documentation, and the Software Requirements Specification.

\subsection{Problem Reporting}

Any problems discovered must be reported to the Test Team Leader, then
eventually the project Administrative Lead, to be fixed.

\subsection{Progress Reporting}

Following the test procedure, the Test Team Leader will compile a report to
submit to the Administrative Lead.

\section{Testing Project Test Cases}

\subsection{Introduction} % AG
	The test cases are divided into sections covering parts of the functionality and use cases in the Software Requirements Specification. Each test case has the following format:
	\begin{itemize}
	  \item Name - The name of the test case
	  \item Preconditions - Conditions needed to initiate the test case
	  \item Actions - The actions expected from a tester
	  \item Post conditions - The expected outcome of the test case
	\end{itemize}

\subsection{Test Cases}
\subsubsection{User Account Creation} % CR form validation 

\begin{testcase}
  Preconditions  & The user visits the Mashbot registration page in a web browser. \\ \hline
  Actions              & The tester enters a username, password, and other relevant information in the appropriate fields and submits for registration. \\ \hline
  Postconditions & The tester is registered in the system if all field verifications are successful. \newline or \newline The tester is not registered in the system if any field verification is not successful. \\ \hline
\end{testcase}

\subsubsection{User Account Modification} % CR

\begin{testcase}
  Preconditions  & The user visits their profile / settings page in a web browser. \\ \hline
  Actions              & The tester enters updated information in any field, except for username, and submits for update. \\ \hline
  Postconditions & The tester is registered in the system if all field verifications are successful. \newline or \newline The tester is not registered in the system if any field verification is not successful. \\ \hline
\end{testcase}

\subsubsection{Logging In} % CR

\begin{testcase}
  Preconditions  & The user has visited the Mashbot web page in a web browser. \\ \hline
  Actions              & The tester enters his username and password in the appropriate fields and submits for verification. \\ \hline
  Postconditions & The tester is logged in to the system if the verification is successful. \newline or \newline The tester is not logged in to the system if the verification is not successful. \\ \hline
\end{testcase}

\subsubsection{Logging Out} % CR

\begin{testcase}
  Preconditions  & The tester is logged into the Mashbot web application. \\ \hline
  Actions              & The tester clicks the logout button. \\ \hline
  Postconditions & The tester is logged out of the system and their session is destroyed. \\ \hline
\end{testcase}

\subsubsection{Login Timeout} % CR

\begin{testcase}
  Preconditions  & The tester is logged into the Mashbot web application. \\ \hline
  Actions              & The tester does not take any action for a predefined amount of time. \\ \hline
  Postconditions & The tester is logged out of the system and their session is destroyed. \\ \hline
\end{testcase}

\subsubsection{Campaign Creation} % JK
\begin{testcase}
  Preconditions  & User is logged into the Mashbot web application. \\ \hline
  Action         & User selects ``Create'' tab.  \\ \hline
  Action         & User fills out all fields presented on the ``Create'' view. \\ \hline
  Action         & User clicks submit button. \\ \hline
  Postconditions & New campaign is created. \\ \hline
\end{testcase}
\subsubsection{Campaign Modification} % JK
\begin{testcase}
  Precondition  & User is logged into the Mashbot web application \\ \hline
  Precondition  & At least one campaign has been created. \\ \hline  
  Precondition  & User is viewing the list of all campaigns.  \\ \hline  
  Action        & User clicks to edit the campaign.  \\ \hline  
  Reaction      & User is redirected to the ``Edit'' view for a campaign.  \\ \hline
  Action        & User changes fields.  \\ \hline  
  Action        & User presses ``Save'' button.  \\ \hline  
  Postcondition & Campaign is updated with changed fields. \\ \hline
  Postcondition & User is taken to view list of all campaigns \\ \hline
\end{testcase}
\subsubsection{Add Campaign Content} % JK
\begin{testcase}
  Precondition  & User is logged into the Mashbot web application  \\ \hline  
  Precondition  & User is viewing list of campaigns.   \\ \hline  
  Action        & User selects a campaign to which to add content. \\ \hline  
  Reaction      & User is taken to ``Add Content'' view. \\ \hline
  Action        & User clicks ``Add New Content'' button. \\ \hline
  Reaction      & A view to select content type appears. \\ \hline
  Action        & User selects content type. \\ \hline
  Reaction      & User is taken to page to edit content for given content type. \\ \hline
  Action        & User completes content. \\ \hline
  Postcondition & Content is added to the campaign. \\ \hline
  Postcondition & User is taken to the ``Add Content'' view. \\ \hline
\end{testcase}
\subsubsection{Modify Campaign Content} % JK
\begin{testcase}
  Precondition  & User is logged into the Mashbot web application. \\ \hline
  Precondition  & User is viewing a campaign. \\ \hline
  Action        & User clicks on a piece of content to edit. \\ \hline
  Reaction      & User is given the ``Edit Content'' view. \\ \hline
  Action        & User completes changes. \\ \hline
  Action        & User clicks the save button. \\ \hline
  Postcondition & User is directed back to the Campaign view. \\ \hline
  Postcondition & Content is saved. \\ \hline
\end{testcase}
\subsubsection{Content Scheduler} % JK
\subsubsubsection{Schedule Content}

\begin{testcase}
  Precondition  & User is logged in to the Mashbot web application. \\ \hline
  Precondition  & User is on the ``Schedule'' view. \\ \hline
  Action        & User clicks and drags a piece of content from the Content Bucket to the calendar. \\ \hline
  Postcondition & Content is scheduled for the date the date at which the user dropped the content. \\ \hline
  Postcondition & A view is given to add the time for the date for which the content is scheduled. \\ \hline
\end{testcase}
\subsubsubsection{Unschedule Content}

\begin{testcase}
  Precondition  & User is logged in to the Mashbot web application. \\ \hline
  Precondition  & User is on the ``Schedule'' view. \\ \hline
  Precondition  & At least one piece of content is already scheduled. \\ \hline
  Action        & User clicks and drags scheduled content from view to Content Bucket. \\ \hline
  Postcondition & Content is removed from the calendar and is unscheduled. \\ \hline
\end{testcase}
\subsubsubsection{Edit Scheduled Time}

\begin{testcase}
  Precondition  & User is logged in to the Mashbot web application. \\ \hline
  Precondition  & User is on the ``Schedule'' view. \\ \hline
  Action        & User clicks a piece of content on the Calendar. \\ \hline
  Postcondition & User is given the ``Edit Scheduled Time'' view. \\ \hline
\end{testcase}
\subsubsubsection{Edit Scheduled Date}

\begin{testcase}
  Precondition  & User is logged in to the Mashbot web application. \\ \hline
  Precondition  & User is on the ``Schedule'' view. \\ \hline
  Precondition  & At least one piece of content is already scheduled. \\ \hline
  Action        & User drags content from one date to another. \\ \hline
  Postcondition & Content's date is changed to date on which the content is dropped. \\ \hline
\end{testcase}
\subsubsection{Content Publisher} % JK
\begin{testcase}
  Precondition  & Content is scheduled to be published. \\ \hline
  Action        & Scheduled time of publishing occurs. \\ \hline
  Postcondition & Content is published to the service it was scheduled to be published to, and at the time it was scheduled for. \\ \hline
\end{testcase}
\subsubsection{Adding External Service Account} % JK
\begin{testcase}
  Precondition  & User is logged in and viewing the External Service Account control panel. \\ \hline
  Action        & User clicks ``Add Account''. \\ \hline
  Reaction      & User is given fields to authenticate the account. \newline or \newline User is redirected to an authentication page on the external service itself. \\ \hline
  Action        & User fills in account credentials in provided fields. \newline or \newline User affirms authentication on authentication page. \\ \hline
  Postcondition & User is approved to use the account. \\ \hline
  Postcondition & Authentication credentials are stored for future use. \\ \hline
\end{testcase}
\end{document}


% LocalWords:  Mashbot JK LocalWords
