\documentclass{article}

\usepackage{titlesec}

\title{Launch Report - Cycle 1}
\author{Team Mashbot!}

\titleformat{\section}[block]
{\normalfont\fillast}
{\scshape section}
{}
{\eugH
 \titlerule[1pt]}

\titleformat{\subsection}[display]
{\normalfont\fillast}
{\scshape section}
{}
{\LARGE}

\begin{document}
\maketitle
\thispagestyle{empty}
\vfill
\begin{tabbing}
\noindent
\textbf{Instructor:} \hspace{3em} \= Lee Leitner \\*[.5em]
\textbf{Team Members:} \> Andrew Gall \\*
\> G. Nick D'Andrea \\*
\> Cody Ray \\*
\> Vito Salerno \\*
\> Josiah Kiehl \\*[.5em]
\textbf{Cycle:} \> 1  \\*[.5em]
\textbf{Date Submitted:} \> \today  \\
\end{tabbing}
\cleardoublepage

\section*{Launch Report - Cycle 1}

This report documents the initial launch of the project.  It includes the product and team names, and an overview description of the product.  It also includes a summary of issues related to the product and to the team, and basic elements of a plan for cycle 1.

\section*{The Product}

\paragraph*{Product Name:} Mashbot!

\subsection*{Overview}

A service that connects to a set of social APIs and provides a unified
interface to push and pull data, using a plugin style architecture so
that support for arbitrary services can be added without changing
existing code.  This would be a core on which applications could be
built.  For instance, the application we intend to build on top of
this core is a tool for small and medium sized businesses to manage
social networking campaigns. Features include scheduling publish dates
for pieces of content, coordinated across services, as well as
allowing multiple users to manage "official" accounts.  Metrics would
also be generated, including information on clickthrough/conversion
rates per link published, useful for tracking the success of a
campaign.

\subsection*{Feature Highlights}

\subsection*{Sponsor, Advisor or Proxy User}

\subsection*{Risks and Issues}
\subsection*{The Project Team}
\paragraph{Team Name:} Team Mashbot!
\paragraph{Team Members and Roles:} \hspace*{\fill} \\*
\begin{tabular}{|l|l|l|l|}
\hline
Name & Major & Role & Comments \\ \hline
Andrew Gall & Software Engineering & Design, Scope, Implement, Test & a.k.a Peaches \\ \hline
G. Nick D'Andrea & Computer Science & \hspace*{\fill}''\hspace*{\fill} &  \\ \hline
Cody Ray & Electrical Engineering & \hspace*{\fill}''\hspace*{\fill} &  \\ \hline
Vito Salerno & Computer Science & \hspace*{\fill}''\hspace*{\fill} &  \\ \hline
Josiah Kiehl & Software Engineering & \hspace*{\fill}''\hspace*{\fill} & \\ \hline
\end{tabular}

\paragraph{Team Communication}
\paragraph{Team Issues}

\subsection*{Project Planning}
\subsection*{Objectives}
\subsection*{Schedule}
\subsection*{Current Status}

Indicate what is currently completed. This should tie into the actuals information in your plan. Be sure to file actual time spent, and be able to compare estimates with actuals.




\end{document}
