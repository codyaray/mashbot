\documentclass{article}

\usepackage{titlesec}
\usepackage{hyperref}

\title{Launch Report - Cycle 1}
\author{Team Mashbot!}

\newcommand{\subsecformat}[1]{\LARGE\MakeLowercase{#1}}
\newcommand{\secformat}[1]{\Huge\MakeLowercase{#1}}
% \so spaces out letters
\titleformat{\section}[block]
{\normalfont\scshape\filcenter}
{\thesection}
{1em}
{\secformat }

\titleformat{\subsection}[block]
{\normalfont\scshape\filright}
{\thesection}
{1em}
{\subsecformat }

\begin{document}
\maketitle
\thispagestyle{empty}
\vfill
\begin{tabbing}
\noindent
\textbf{Instructor:} \hspace{3em} \= Lee Leitner \\*[.5em]
\textbf{Team Members:} \> Andrew Gall \\*
\> G. Nick D'Andrea \\*
\> Cody Ray \\*
\> Vito Salerno \\*
\> Josiah Kiehl \\*[.5em]
\textbf{Cycle:} \> 1  \\*[.5em]
\textbf{Date Submitted:} \> \today  \\
\end{tabbing}
\cleardoublepage

\section*{Launch Report - Cycle 1}

This report documents the initial launch of the project.  It includes the product and team names, and an overview description of the product.  It also includes a summary of issues related to the product and to the team, and basic elements of a plan for cycle 1.

\section*{The Product}

\paragraph*{Product Name:} Mashbot!

\subsection*{Overview}

A service that connects to a set of social APIs and provides a unified
interface to push and pull data, using a plugin style architecture so
that support for arbitrary services can be added without changing
existing code.  This would be a core on which applications could be
built.  For instance, the application we intend to build on top of
this core is a tool for small and medium sized businesses to manage
social networking campaigns. Features include scheduling publish dates
for pieces of content, coordinated across services, as well as
allowing multiple users to manage ``official'' accounts.  Metrics would
also be generated, including information on clickthrough/conversion
rates per link published, useful for tracking the success of a
campaign.

\subsection*{Feature Highlights}
\begin{itemize}
  \item Full featured scheduling utility to allow users to plan far in advance their tweets, blog posts, image and video uploads, and status updates.
  \item Extensible core that allows plugins for additional service support to be written in the future.
  \item Full copy editing and approval process to adhere to common business practices with regard to marketing workflow.
  \item CRUD support for all user owned content (where allowed by APIs)
  \item Insight dashboard that provides a view on what is being talked about in social media, especially with regard to the business user's domain.
\end{itemize}

\subsection*{Sponsor, Advisor or Proxy User}

\textbf{Sponsor:} No outside funding \\*
\textbf{Advisor:} Jeffry Popyack \\*
\textbf{Proxy User:} \\*

\subsection*{Risks and Issues}

\begin{itemize}
  \item Lack of understanding of the problem space: none of the developers are in marketing directly.
  \item Underscoping the project causing not all goals to be met.
  \item Dealing with the old-school style (waterfall style) development that is dictated by the Senior Design plan.
\end{itemize}

\subsection*{The Project Team}
\paragraph{Team Name:} Team Mashbot!
\paragraph{Team Members and Roles:} \hspace*{\fill} \\*
\begin{tabular}{|l|l|l|l|}
\hline
Name & Major & Role & Comments \\ \hline
Andrew Gall & Software Engineering & Design, Scope, Implement, Test & a.k.a Peaches \\ \hline
G. Nick D'Andrea & Computer Science & \hspace*{\fill}''\hspace*{\fill} &  \\ \hline
Cody Ray & Electrical Engineering & \hspace*{\fill}''\hspace*{\fill} &  \\ \hline
Vito Salerno & Computer Science & \hspace*{\fill}''\hspace*{\fill} &  \\ \hline
Josiah Kiehl & Software Engineering & \hspace*{\fill}''\hspace*{\fill} & \\ \hline
\end{tabular}

\paragraph{Team Communication}
\begin{itemize}
  \item Google Group (List Serv)
  \item Github Wiki
  \item gChat Instant Messaging
\end{itemize}
\paragraph{Team Issues}

None identified at this point.

\section*{Project Planning}

\subsection*{Objectives}

\begin{itemize}
  \item To provide a tool that minimizes the time commitment for a small business to utilize free services on the internet for marketing and online persona purposes.
  \item To expose a core API for web services that will allow developers to add support for new content aggregation without altering the existing code base.
  \item To build a platform that will support application agnostic support for webservice APIs, allowing developers to work with concepts of content genres rather than dealing with each individual service's content individually.
\end{itemize}
\subsection*{Schedule}

See the attached ProjectPlan.html

\subsection*{Current Status}

User Stories have been created, technology has been selected, planning is underway.




\end{document}
