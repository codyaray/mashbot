\documentclass{article}
\begin{document}

\title{Software Requirements Specification for Mashbot} 
\author{George D'Andrea \and \and Andrew Gall \and Josiah Kiehl \and
  Cody Ray \and Vito Salerno}
\date{\today}
\begin{titlepage}
\maketitle
\end{titlepage}

\tableofcontents

\section{Introduction}

\subsection{Purpose}
\subsection{Scope}
\subsection{Glossary}
\subsection{Overview}

\section{Overall Description}
	\subsection{Product Perspective}
		For many startups and small businesses, marketing /
                customer service and business / development efforts
                compete for scarce resources.  Furthermore, early
                stage startups often have small or non-existent
                marketing budgets, especially in resource-hungry
                product-based startups.  Although the plethora of
                widely available and affordable communications
                technology (e.g., social media, VoIP, streaming video,
                etc.) theoretically reduces both capital and operating
                expenditures related to marketing, the quantity of
                such technologies and services effectively creates an
                "opportunity overload" making the process of
                maintaining a focused and effective marketing campaign
                more difficult.  Furthermore, with the number of
                services, technologies, and "specialists" arriving
                each day, its nearly impossible to keep up with all
                the trends much less to fully take advantage of each
                or even monitor them all adequately! As you can see,
                the widespread adoption of communications technology
                is a double-edged sword---there are many more avenues
                for marketing and customer service, but it is much
                more difficult to effectively apply reputation
                management strategies or adequate customer service
                through all channels. This project proposes to build a
                tool to increase the effectiveness and efficiency of
                marketing campaigns and customer service for small to
                medium businesses.
		\subsubsection{System Interfaces}
			Mashbot combines several components to provide the functionality
			required.
			\begin{itemize}
				\item[Authentication]
				\item[Campaign Manager Web Interface]
				\item[Core!]
				\item[Database]
			\end{itemize}
		\subsubsection{User Interface}
			The user interface consists of a web front end
			with tabs to separate the various workflow areas. To
			create content, the user is provided with a calendar
			scheduling tool, and a content editor.  Additionally,
			there is a monitoring dashboard which gives the user a
			view on responses to the content in any given campaign.
		\subsubsection{Hardware Interfaces}
                It's a freaking web application.
		\subsubsection{Software Interfaces}
                It's a freaking web application.
		\subsubsection{Communications Interfaces}
                It's a freaking web application.
		\subsubsection{Memory Constraints}
                Server must not crash.
		\subsubsection{Site Adaptation Requirements}
                TODO: Ask Popyack if needed.
	\subsection{Product Functions}
        Mashbot should be able to:
        \begin{enumerate}
          \item Schedule content for various services to be
            published concurrently
          \item View and Compare historical metrics of campaigns
          \item View/Create replies to content
          \item Maintain users and approvers of content
          \item Set up keyword alerts for ``watched'' services
        \end{enumerate}
	\subsection{User Characteristics}
        The target user is a small to medium business employee who understands
        the basic capabilities of social media.
        
	\subsection{Requirements Apportioning}
        \begin{tabular}{|c|p{4in}|}
          \hline
          \textbf{Priority} & \textbf{Description} \\
          \hline \hline
          1 & Mashbot can not be released unless it satisfies these
              requirements. \\
          \hline
          2 & Mashbot may be initially released without satisfying these
              requirements. Having these requirements unfulfilled must
              not create dangers to the system. They should be implemented in the next
              minor release. \\
          \hline
          3 & These requirements are not expected to be fulfilled by
              the initial release of Mashbot, but should be
              implemented in the next major release. \\
          \hline
          4 & These requirements are outside the current scope of the
              project, but are included to exhibit how our software
              will improve in the future. \\
          \hline
        \end{tabular}

\section{Specific Requirements}
	\subsection{External Interface Requirements}
	\subsection{Functional Requirements}
		\subsubsection{User Accounts}
			\begin{itemize}
				\item \bf{User Account Types and Permissions} The system categorizes users on the basis
				of roles and privileges. Within these roles, the system also categorizes users based on
			 	the roles that they have within individual products.	These are referred to as user 
				account roles.
				\begin{itemize}
					\item Mashbot! Campaigns supports the following account roles:
						\begin{enumerate}
							\item Content Approver
							\item Content Creator
							\item Content Editor
							\item Content Release Scheduler
							\item Content Responder
							\item Product Administrator
							\item System Administrator
						\end{enumerate}
					\item A user may possess more than one role.
					\item Roles reflect actions that can be performed by a user.
					\item Roles can be assigned to a user account for individual products.
					\item \bf{Content Approver} can approve for release content which has been
					submitted for approval
					\item \bf{Content Creators} may create new content or import existing
					content into the system. They may also submit this content for approval.
					\item \bf{Content Editor} may edit content which has already been posted to external
					services.
					\item \bf{Content Release Scheduler} may schedule or immediately initiate the release 
					to external services of content which has already been submitted and approved.
					\item \bf{Content Responder} may respond to responses which have been made to the posted 
					content within the external services.
				 	\item \bf{Product Administrator} is responsible for setting individual roles for a product.
						\begin{itemize}
							\item There must be at least one product administrator for each product.
						\end{itemize}	
					\item \bf{System Administrator} can perform any of the roles in the system for any product.
					Additionally, they can set any system-wide settings which effect all products in the system.
				\end{itemize}
			\end{itemize}

		\subsubsection{Products}
			\begin{itemize}
				\item \bf{Product Types} The system categorizes products based on the type of content they push.
			 		These are referred to as product content types.
				\begin{itemize}
					\item Mashbot! Campaigns support the following product content types:
						\begin{enumerate}
							\item Blog
							\item Pictures
							\item Status
							\item Video
						\end{enumerate}
					\item A product may possess more than one content type.
					\item \bf{Blog} may be added, removed, edited, and replies may be collected and responded to. 
					\item \bf{Pictures} may be added, removed, and replies may be collected and responded to.
					\item \bf{Status} may be added, removed, edited, and replies may be collected and responded to. 
					\item \bf{Video} may be added, removed, and replies may be collected and responded to.
				\end{itemize}	
			\end{itemize}
		\subsubsection{Marketing Campaigns}
		\subsubsection{External Service Accounts}

	\subsection{Security Requirements}
	\subsection{Design Constraints}
	\subsection{Software System Attributes}
	\subsection{User Interface}

\section{Preliminary Analysis}

\section{Use Cases}

\section{IO}

Inputs
  CRUD
  Types
    Pictures
    Status/Tweet
    Blog
    Video

Users
  Name
  Email
  Company
  Permissions
  Accounts
    
Account
  What a company buys
  Allows user/campaign creation
  
Campaigns
  Name
  Schedule
  Content
    Distribution Channels


\end{document}


% LocalWords:  Mashbot
