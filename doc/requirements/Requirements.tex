\documentclass{article}
\begin{document}

\section{Introduction}

\subsection{Purpose}
\subsection{Scope}
\subsection{Glossary}
\subsection{Overview}

\section{Overall Description}
	\subsection{Product Perspective}
		For many startups and small businesses, marketing / customer service
		and business / development efforts compete for scarce resources.
		Furthermore, early stage startups often have small or non-existent
		marketing budgets, especially in resource-hungry product-based
		startups.  Although the plethora of widely available and affordable
		communications technology (e.g., social media, VoIP, streaming video,
			etc.) theoretically reduces both capital and operating
		expenditures related to marketing, the quantity of such technologies
		and services effectively creates an "opportunity overload" making the
		process of maintaining a focused and effective marketing campaign more
		difficult.  Furthermore, with the number of services, technologies,
		and "specialists" arriving each day, its nearly impossible to keep up
		with all the trends much less to fully take advantage of each or even
		monitor them all adequately! As you can see, the widespread adoption of communications technology is a double-edged sword---there are many more avenues for marketing and customer service, but it is much more difficult to effectively apply reputation management strategies or adequate customer service through all channels. This project proposes to build a tool to increase the effectiveness and efficiency of marketing campaigns and customer service for small to medium businesses.  
		\subsubsection{System Interfaces}
			Mashbot combines several components to provide the functionality
			required.
			\begin{itemize}
				\item[Authentication]
				\item[Campaign Manager Web Interface]
				\item[Core!]
				\item[Database]
			\end{itemize}
		\subsubsection{User Interface}
                The user interface consists of a web front end
                with tabs to separate the various workflow areas. To
                create content, the user is provided with a calendar
                scheduling tool, and a content editor.  Additionally,
                there is a monitoring dashboard which gives the user a
                view on responses to the content in any given campaign.
		\subsubsection{Hardware Interfaces}
                It's a freaking web application.
		\subsubsection{Software Interfaces}
                It's a freaking web application.
		\subsubsection{Communications Interfaces}
                It's a freaking web application.
		\subsubsection{Memory Constraints}
                Server must not crash.
		\subsubsection{Site Adaptation Requirements}
                TODO: Ask Popyack if needed.
	\subsection{Product Functions}
        Mashbot should be able to:
        \begin{enumerate}
          \item Schedule content for various services to be
            published concurrently
          \item View and Compare historical metrics of campaigns
          \item View/Create replies to content
          \item Maintain users and approvers of content
          \item Set up keyword alerts for ``watched'' services
        \end{enumerate}
	\subsection{User Characteristics}
        The target user is a small to medium business employee who understands
        the basic capabilities of social media.
        
	\subsection{Requirements Apportioning}

\section{Specific Requirements}
	\subsection{External Interface Requirements}
	\subsection{Functional Requirements}
	\subsection{Security Requirements}
	\subsection{Design Constraints}
	\subsection{Software System Attributes}
	\subsection{User Interface}

\section{Preliminary Analysis}

\section{Use Cases}

\section{IO}

Inputs
  CRUD
  Types
    Pictures
    Status/Tweet
    Blog
    Video

Users
  Name
  Email
  Company
  Permissions
  Accounts
    
Account
  What a company buys
  Allows user/campaign creation
  
Campaigns
  Name
  Schedule
  Content
    Distribution Channels


\end{document}

