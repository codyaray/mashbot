\documentclass[12pt]{article}
\usepackage{fullpage}
\usepackage[parfill]{parskip}    % Activate to begin paragraphs with an empty line rather than an indent

\title{Integration Test Plan for Mashbot} 

\author{George D'Andrea \and Andrew Gall \and Josiah Kiehl \and
  Cody Ray \and Vito Salerno}
\date{\today}

\begin{document}
\maketitle

\section{Introduction}
\subsection{Background}
This document provides the high level testing requirements for the Mashbot Campaign Manager and Publishing and Aggregation Platforms.  These tests will be implemented as behavioral tests that will be automatically run every time a feature is added to prevent regression.
\subsection{References}
\begin{itemize}
  \item Design | 
\end{itemize}
\subsection{Definitions and Acronyms}

\section{Approach and Constraints}
\subsection{Objectives}
\subsection{Structure}
\subsection{Constraints}

\section{Assumptions and Exclusions}
\subsection{Assumptions}
\subsection{Exclusions}

\section{Entry and Exit Criteria}
\subsection{Entry Criteria}
\subsection{Exit Criteria}

\section{Testing Participants}
\subsection{Roles and Responsibilities}
\subsection{Training Requirements}
\subsection{Problem Reporting}
\subsection{Progress Reporting}

\section{Test Cases}

\subsection{Core}

\subsubsection{Feature: Push Status}

Feature: Push Status \\
	Push a status to multiple services \\
	As a client \\
	I want to be able to get the ids for each individual status post to each  
	service that I specify. \\
	
Scenario: Pushing a status to Twitter \\
	Given that I have sent a request \\
	And the authentication credentials are valid \\
	And I have filled in at least the status field in the object model \\
	And I have provided authentication credentials for Twitter \\ 
	And Those credentials are valid for Twitter \\ 
	And I have indicated an association with the service Twitter \\
	Then there will be a status posted on behalf of the user whose credentials 
	were provided to Twitter \\

Scenario: Pushing a status to Twitter and Facebook \\
	Given that I have sent a request \\
	And the authentication credentials are valid \\
	And I have filled in at least the status field in the object model \\
	And I have provided authentication credentials for Twitter and Facebook \\ 
	And Those credentials are valid for Twitter and Facebook \\ 
	And I have indicated an association with the service Twitter and Facebook \\
	Then there will be a status posted on behalf of the user whose credentials 
	were provided to Twitter and Facebook \\

Scenario: Pushing a status to Twitter and Facebook where the required 
information for one or both is not present. \\

Scenario: Pushing a status to Twitter and Facebook where one or both of sets of 
authentication credentials are invalid. \\

Scenario: Pushing a status to Twitter and Facebook where one or both services 
reject our request. \\


\subsubsection{Feature: Pull Status}
\subsubsection{Feature: Edit Status}
\subsubsection{Feature: Delete Status}

\subsubsection{Feature: Push Blog}
Feature: Push Blog \\
	Push a blog post to multiple services \\
	As a client \\
	I want to be able to get the ids for each individual blog post to each  
	service that I specify. \\
	
Scenario: Pushing a blog post to Tumblr \\
	Given that I have sent a request \\
	And the authentication credentials are valid \\
	And I have filled in at least the blog post field in the object model \\
	And I have provided authentication credentials for Tumblr \\ 
	And Those credentials are valid for Tumblr \\ 
	And I have indicated an association with the service Tumblr \\
	Then there will be a blog post posted on behalf of the user whose credentials 
	were provided to Tumblr \\

Scenario: Pushing a blog post to Tumblr and Wordpress \\
	Given that I have sent a request \\
	And the authentication credentials are valid \\
	And I have filled in at least the blog post field in the object model \\
	And I have provided authentication credentials for Tumblr and Wordpress \\ 
	And Those credentials are valid for Tumblr and Wordpress \\ 
	And I have indicated an association with the service Tumblr and Wordpress \\
	Then there will be a blog post posted on behalf of the user whose credentials 
	were provided to Tumblr and Wordpress \\

\subsubsection{Feature: Pull Blog}
\subsubsection{Feature: Edit Blog}
\subsubsection{Feature: Delete Blog}

\subsubsection{Feature: Push Picture}
Feature: Push Picture \\
	Push a picture to multiple services \\
	As a client \\
	I want to be able to get the ids for each individual picture to each  
	service that I specify. \\
	
Scenario: Pushing a picture to Flickr \\
	Given that I have sent a request \\
	And the authentication credentials are valid \\
	And I have filled in at least the picture field in the object model \\
	And I have provided authentication credentials for Flickr \\ 
	And Those credentials are valid for Flickr \\ 
	And I have indicated an association with the service Flickr \\
	Then there will be a picture posted on behalf of the user whose credentials 
	were provided to Flickr \\

Scenario: Pushing a picture to Flickr and Picasa \\
	Given that I have sent a request \\
	And the authentication credentials are valid \\
	And I have filled in at least the picture field in the object model \\
	And I have provided authentication credentials for Tumblr and Wordpress \\ 
	And Those credentials are valid for Flickr and Picasa \\ 
	And I have indicated an association with the service Flickr and Picasa \\
	Then there will be a picture posted on behalf of the user whose credentials 
	were provided to Flickr and Picasa \\
\subsubsection{Feature: Pull Picture}
\subsubsection{Feature: Edit Picture}
\subsubsection{Feature: Delete Picture}

\subsubsection{Feature: Push Video}
\subsubsection{Feature: Pull Video}
\subsubsection{Feature: Edit Video}
\subsubsection{Feature: Delete Video}

\subsection{Campaign Manager}

\subsubsection{Feature: Login}

Feature: Login \\
  In order to be able to log in \\
  As a user \\
  I want to be able to see the log in link in the header when I am not logged in. \\

Scenario: Viewing the right header links while not logged in  \\
  Given I am on the home page \\
  Then I should see ``Login'' within ``\#administrative-links'' \\
  And I should see ``Register'' within ``\#administrative-links'' \\
  And I should not see ``Logout'' within ``\#administrative-links'' \\

Scenario: Registering a new account \\
  Given I am on the home page \\
  And there is no user using the email address ``bloo@example.net'' \\
  And I follow ``Register'' \\
  And I fill in ``Bloo'' for ``Login'' \\
  And I fill in ``bloo@example.net'' for ``Email'' \\
  And I fill in ``bloospassword'' for ``Password'' \\
  And I fill in ``bloospassword'' for ``Password confirmation'' \\
  And I fill in ``Bloo Corp'' for ``Company'' \\
  And I press ``Register'' \\
  Then there should be a user with the email ``bloo@example.net'' and the username ``bloo'' from the company ``Bloo Corp'' \\

Scenario: Logging In \\
  Given I am on the home page \\
  And there is a user named ``pojo'' with the email ``example@pojo.com'', with the password ``bumble\_bee1'' from the company ``Bloo Corp'' \\
  When I follow ``Login'' within ``\#administrative-links'' \\
  And I fill in ``pojo'' for ``Login'' \\
  And I fill in ``bumble\_bee1'' for ``Password'' \\
  And I press ``Login'' \\
  Then I should see ``pojo'' within ``\#administrative-links'' \\
  And I should see ``Logout'' within ``\#administrative-links'' \\
  And I should see ``My Account'' within ``\#administrative-links'' \\
  And I should see ``Bloo Corp'' within ``.title'' \\

Scenario: Wrong Password \\
  Given I am on the home page \\
  And there is a user named ``pojo'' with the email ``example@pojo.com'', with the password ``bumble\_bee1'' from the company ``Bloo Corp'' \\
  When I follow ``Login'' within ``\#administrative-links'' \\
  And I fill in ``pojo'' for ``Login'' \\
  And I fill in ``omgwrongpassword'' for ``Password'' \\
  And I press ``Login'' \\
  Then I should see ``Password is not valid'' \\

Scenario: Wrong Login \\
  Given I am on the home page \\
  And there is a user named ``pojo'' with the email ``example@pojo.com'', with the password ``bumble\_bee1'' from the company ``Bloo Corp'' \\
  When I follow ``Login'' within ``\#administrative-links'' \\
  And I fill in ``notpojo'' for ``Login'' \\
  And I fill in ``bumble\_bee1'' for ``Password'' \\
  And I press ``Login'' \\
  Then I should see ``Login is not valid'' \\

\subsubsection{Feature: Create Campaign}

Feature: Create Campaign \\
  In order to create a campaign \\
  As a user \\
  I want to create campaigns that will be able to hold content for publishing. \\
  
Scenario: Creating a campaign \\
  Given pojo is logged in \\
  When I go to the create campaign page \\
  And I fill in ``Amazing Campaign'' for ``Title'' \\
  And I fill in ``01/15/2100'' for ``Start date'' \\
  And I fill in ``02/14/2100'' for ``End date''   \\
  And I press ``Create!'' \\
  Then I should see ``Amazing Campaign'' \\
  And I should see ``01/15/2100'' \\
  And I should see ``02/14/2100'' \\

Scenario: Creating a campaign \\
  Given pojo is logged in \\
  When I go to the create campaign page \\
  And I fill in ``Unfail Campaign'' for ``Title'' \\
  And I press ``Create!'' \\
  Then I should see ``Unfail Campaign'' \\

Scenario: Validate sane start and end dates: start should be before end. \\
  Given pojo is logged in \\
  When I go to the create campaign page \\
  And I fill in ``Fail Campaign'' for ``Title'' \\
  And I fill in ``01/15/2100'' for ``Start date'' \\
  And I fill in ``01/14/2100'' for ``End date'' \\
  And I press ``Create!'' \\
  Then I should see ``Hold up! You can't end something before you start it!'' \\

Scenario: Make sure you can't have an end date without a start date. \\
  Given pojo is logged in \\
  When I go to the create campaign page \\
  And I fill in ``Fail Campaign'' for ``Title'' \\
  And I fill in ``01/14/2100'' for ``End date'' \\
  And I press ``Create!'' \\
  Then I should see ``Whoa buddy! You're going to need a start date if you intend to have an end date'' \\

\subsubsection{Feature: Schedule Campaign}

Feature: Schedule Campaign \\
  In order to schedule campaigns \\
  As a user \\
  I want to schedule and manage the start and end time of campaigns. \\

Scenario: Unscheduled Campaign List \\
  Given pojo is logged in \\
  And I have campaigns titled Twitterblast, Social Network Blitz \\
  When I go to the schedule page \\
  Then I should see ``Twitterblast'' \\
  And I should see ``Social Network Blitz'' \\

\end{document}
